% Created 2022-06-07 Tue 18:52
% Intended LaTeX compiler: pdflatex
\documentclass[letterpaper, 12pt]{report}
\usepackage[utf8]{inputenc}
\usepackage[T1]{fontenc}
\usepackage{graphicx}
\usepackage{longtable}
\usepackage{wrapfig}
\usepackage{rotating}
\usepackage[normalem]{ulem}
\usepackage{amsmath}
\usepackage{amssymb}
\usepackage{capt-of}
\usepackage{hyperref}
\usepackage{amsmath}		% Extra math definitions
\usepackage{graphics}		% PostScript figures
\usepackage{setspace}		% 1.5 spacing
\usepackage{longtable}          % Tables spanning pages
\usepackage{natbib}
\usepackage{times}
\usepackage{url}
\usepackage{latexsym}
\usepackage[usenames]{color}
\usepackage{covington}
\usepackage{graphicx}
\usepackage{multirow}
\usepackage{subcaption}%#+LATEX_HEADER: \usepackage{subfigure}
\usepackage{booktabs}
\usepackage{tabularx}
\usepackage[T1]{fontenc}
\usepackage[utf8]{inputenc}
\usepackage[english]{babel}
\usepackage{blindtext}
\usepackage{amsfonts}
\usepackage{amsthm}
\usepackage[table,xcdraw]{xcolor}
\usepackage{rotating}
\usepackage{listings}
\definecolor{NiceBlue}{RGB}{11, 102, 163}
\definecolor{SlightRed}{RGB}{249,38,114}
\usepackage{textcomp} % other glyphs needed for upquote in listings below
\lstdefinelanguage{DemoExample}
{ basicstyle=\footnotesize \ttfamily,
commentstyle=\color{SlightRed} \rmfamily\itshape,
stringstyle=\color{NiceBlue},
morecomment=[s]{/*}{*/},
morestring=[b]'
}
\usepackage[fancyhdr]{macros/McECEThesis}	% Thesis style
\usepackage{McGillLogo}		% McGill University crest
\usepackage{color}
\insidemargin = 1.1in
\outsidemargin = 1.1in
\abovemargin = 1.1in
\belowmargin = 0.75in
\newcommand{\beq}{\begin{equation}}
\newcommand{\eeq}{\end{equation}}
\usepackage{palatino}           % Less abusive fonts
\usepackage{macros/palatcm}
\usepackage{hyperref}
\let\mathexp=\exp %redefine \exp to \mathexp cuz gb4e package redefines \exp
\usepackage{gb4e}
\noautomath
\usepackage[acronym,toc,section=section]{glossaries}
\makeglossaries
\newglossaryentry{tlm}{name=Transformers,description={{A class of models first derived by Vaswani et al. 2017}}}
\newacronym{llm}{LLMs}{Large Language Models}
\newacronym{qos}{QoS}{quality-of-service}
\newacronym{bb}{BB}{branch and bound}
\author{Koustuv Sinha}
\date{}
\title{PhD Thesis}
\hypersetup{
 pdfauthor={Koustuv Sinha},
 pdftitle={PhD Thesis},
 pdfkeywords={},
 pdfsubject={},
 pdfcreator={Emacs 28.1 (Org mode 9.6)}, 
 pdflang={English}}
\begin{document}

\maketitle
\raggedbottom
\spacing{1.5}%\onehalfspacing
\pagenumbering{roman}

\part*{Acknowledgements}
\label{sec:org7e5c0aa}
\part*{Abstract}
\label{sec:org5a975de}
\part*{Abstract in French}
\label{sec:orge2ad76f}
\part*{Contributions to Original Knowledge}
\label{sec:org6846130}
\part*{Contributions of Authors}
\label{sec:org4da2d9f}
\part*{List of Figures}
\label{sec:org165ba31}
\part*{List of Tables}
\label{sec:orgc95eccf}
\clearpage
\setcounter{tocdepth}{3}
\tableofcontents

\clearpage

\pagenumbering{arabic}
\part{Introduction}
\label{sec:org3362b2c}
\textbf{\textbf{Central Theme of the thesis}} : Understanding systematicity in pre-trained language models through semantic and syntactic generalization.


\clearpage
\part{Background}
\label{sec:org92ee0e4}



\chapter{Early methods for text representation}
\label{sec:orgb488c7b}
\chapter{Neural Inductive bias of text representation}
\label{sec:org1bb6be4}
\section{Feed Forward Neural Networks}
\label{sec:org18325e7}
\section{Recurrent Neural Networks}
\label{sec:org01a78da}
\section{Transformer Models}
\label{sec:org7c10c6c}

\gls{llm} are the state-of-the-art in language models, which are based on \gls{tlm}.
\chapter{Pre-training and the advent of Large Language Models}
\label{sec:orgd878583}
Success of pre-training and scale
\chapter{Systematicity and Generalization}
\label{sec:org6500ffe}
\section{Definitions}
\label{sec:org8c6da40}
\begin{enumerate}
\item Productivity
\label{sec:org351593c}
\item Word Order Sensitivity
\label{sec:orgdd85c7b}
\end{enumerate}
\section{Tasks}
\label{sec:org459321e}
\clearpage
\part{Understanding semantic generalization through productivity}
\label{sec:org116f095}

\chapter{Technical Background}
\label{sec:org10c5fec}
\chapter{CLUTRR: A Diagnostic Benchmark for Inductive Reasoning in Text}
\label{sec:org8609c5a}

Paper: \cite{sinha2019a}

\section{Dataset construction}
\label{sec:orgf581075}
\section{Productivity and reasoning}
\label{sec:orgbb2ea61}
\chapter{Results}
\label{sec:org555851f}
\chapter{Discussion}
\label{sec:org4b2a7cb}
\chapter{Follow-up findings in the community}
\label{sec:org2bf2420}
\chapter{Related Work}
\label{sec:orgc3b8019}
\clearpage
\part{Quantifying syntactic generalization using word order}
\label{sec:org3c395f8}

Paper \cite{sinha2021a}

\chapter{Technical Background}
\label{sec:orgb1e99cb}
\chapter{Word Order in Natural Language Inference}
\label{sec:orgd0c45b6}
\section{Probe Construction}
\label{sec:org0b6937c}
\chapter{Experiments \& Results}
\label{sec:org50de332}
\chapter{Discussion}
\label{sec:org6dbee37}
\chapter{Follow-up findings in the community}
\label{sec:org63e0600}
\chapter{Related Work}
\label{sec:org563142f}
\clearpage
\part{Probing syntax understanding through distributional hypothesis}
\label{sec:org0e2525d}

Paper: \cite{sinha2021}

\chapter{Technical Background}
\label{sec:org1c48b12}
\chapter{Dataset construction and pre-training}
\label{sec:org6fccae9}
\chapter{Experiments}
\label{sec:org128fd02}
\section{Downstream reasoning tasks}
\label{sec:org8fc623c}
\section{Evaluating the effectiveness of probing syntax}
\label{sec:org28dddc0}
\chapter{Discussion}
\label{sec:org5425dc6}
\chapter{Follow-up findings in the community}
\label{sec:org0c2fffd}
\chapter{Related Work}
\label{sec:org279d5b6}
\clearpage
\part{Measuring systematic generalization by exploiting absolute positions}
\label{sec:orgf73b150}

\chapter{Technical Background}
\label{sec:org132da13}
\chapter{Systematic understanding of absolute position embeddings}
\label{sec:orgfd0d2b5}
\chapter{Experiments}
\label{sec:orgea347ef}
\chapter{Discussion}
\label{sec:org1bc127e}
\chapter{Related Work}
\label{sec:org75d823e}
\clearpage
\part{Conclusion}
\label{sec:orgcc9c607}
\chapter{Summary}
\label{sec:orgf62566a}
\chapter{Limitations}
\label{sec:org3bdc714}
\chapter{Future Work}
\label{sec:orga1d02ce}
\clearpage
\part{Bibliography}
\label{sec:org2722050}
\bibliographystyle{unsrt}
\bibliography{../../mylibrefs_new}


\printglossaries

\part{Appendix}
\label{sec:org594b8e8}
\chapter{Org mode auto save}
\label{sec:org2a75011}
Run the following snippet to auto save and compile in org mode.

\begin{verbatim}
(defun kdm/org-save-and-export ()
(interactive)
(if (and (eq major-mode 'org-mode)
    (ido-local-file-exists-p (concat (file-name-sans-extension (buffer-name)) ".tex")))
  (org-latex-export-to-pdf)))

(add-hook 'after-save-hook 'kdm/org-save-and-export)
\end{verbatim}
\chapter{Add newpage before a heading}
\label{sec:orgc3c29fd}

\begin{verbatim}
(defun org/get-headline-string-element  (headline backend info)
  (let ((prop-point (next-property-change 0 headline)))
    (if prop-point (plist-get (text-properties-at prop-point headline) :parent))))

(defun org/ensure-latex-clearpage (headline backend info)
  (when (org-export-derived-backend-p backend 'latex)
    (let ((elmnt (org/get-headline-string-element headline backend info)))
      (when (member "newpage" (org-element-property :tags elmnt))
        (concat "\\clearpage\n" headline)))))

(add-to-list 'org-export-filter-headline-functions
             'org/ensure-latex-clearpage)

\end{verbatim}
\chapter{Glossary and Acronym build using Latexmk}
\label{sec:org13c88af}

Add the following snippet in the file ``\textasciitilde{}/.latexmkrc'': (Source: \url{https://tex.stackexchange.com/a/44316})

\begin{verbatim}
add_cus_dep('glo', 'gls', 0, 'run_makeglossaries');
add_cus_dep('acn', 'acr', 0, 'run_makeglossaries');

sub run_makeglossaries {
    my ($base_name, $path) = fileparse( $_[0] ); #handle -outdir param by splitting path and file, ...
    pushd $path; # ... cd-ing into folder first, then running makeglossaries ...

    if ( $silent ) {
        system "makeglossaries -q '$base_name'"; #unix
        # system "makeglossaries", "-q", "$base_name"; #windows
    }
    else {
        system "makeglossaries '$base_name'"; #unix
        # system "makeglossaries", "$base_name"; #windows
    };

    popd; # ... and cd-ing back again
}

push @generated_exts, 'glo', 'gls', 'glg';
push @generated_exts, 'acn', 'acr', 'alg';
$clean_ext .= ' %R.ist %R.xdy';
\end{verbatim}
\end{document}
